\section{Introduction}

Stochastic gradient descent (SGD) has emerged as the cornerstone algorithm for training modern machine learning models, particularly deep neural networks. While its behavior in convex optimization is well-understood \cite{robbins1951}, the theoretical analysis of SGD in non-convex settings remains an active area of research \cite{bottou2018optimization}.

The optimization of non-convex functions presents several fundamental challenges:
\begin{itemize}
    \item Multiple local minima and saddle points can trap optimization algorithms
    \item Noise in gradient estimates can significantly impact convergence
    \item Traditional convex analysis tools do not directly apply
\end{itemize}

Recent work has shown that SGD can effectively escape saddle points \cite{ge2015escaping} and converge to local minima under certain conditions. However, the precise convergence rates and their dependence on problem parameters remain areas of active investigation.

In this paper, we provide new theoretical results on the convergence rates of SGD for non-convex optimization. Our analysis focuses on establishing bounds on the number of iterations required to reach approximate stationary points, with explicit dependencies on the problem parameters and desired accuracy.