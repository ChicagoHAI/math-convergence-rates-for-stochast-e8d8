% Common mathematical notation
% Using providecommand to avoid conflicts with paper-specific macros
\usepackage{amsfonts}

% Number sets
\providecommand{\R}{\mathbb{R}}
\providecommand{\N}{\mathbb{N}}
\providecommand{\Z}{\mathbb{Z}}
\providecommand{\Q}{\mathbb{Q}}
\providecommand{\C}{\mathbb{C}}

% Common operators
\providecommand{\eps}{\varepsilon}
\DeclareMathOperator*{\argmax}{arg\,max}
\DeclareMathOperator*{\argmin}{arg\,min}

% Norms and absolute values
\providecommand{\norm}[1]{\left\|#1\right\|}
\providecommand{\abs}[1]{\left|#1\right|}

% Inner product
\providecommand{\inner}[2]{\langle #1, #2 \rangle}

% Probability and Statistics
\providecommand{\E}{\mathbb{E}}
\providecommand{\Var}{\mathrm{Var}}
\providecommand{\Cov}{\mathrm{Cov}}
\providecommand{\Prob}{\mathbb{P}}
